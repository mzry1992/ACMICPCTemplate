\subsection{精度问题}
	\subsubsection{浮点数为啥会有精度问题}
	浮点数(以C/C++为准),一般用的较多的是float、double。\\
	\begin{table}[htbp]
		\centering
		\begin{tabular}{l|l|l|l}
		\toprule
		&占字节数&数值范围&十进制精度位数\\
		\midrule
		float&$4$&$-3.4e-38\sim 3.4e38$&$6\sim 7$\\
		double&$8$&$-1.7e-308\sim 1.7e308$&$14\sim 15$\\
		\bottomrule
		\end{tabular}
	\end{table}
	~\\
	如果内存不是很紧张或者精度要求不是很低,一般选用double。$14$位的精度(是有效数字位,不是小数点后的位数)通常够用了。注意,问题来了,数据精度位数达到了$14$位,但有些浮点运算的结果精度并达不到这么高,可能准确的结果只有$10\sim 12$位左右。那低几位呢?自然就是不可预料的数字了。这给我们带来这样的问题:即使是理论上相同的值,由于是经过不同的运算过程得到的,他们在低几位有可能(一般来说都是)是不同的。这种现象看似没太大的影响,却会一种运算产生致命的影响:$==$。恩,就是判断相等。注意,C/C++中浮点数的==需要完全一样才能返回true。\\
	
	\subsubsection{eps}
	eps缩写自epsilon,表示一个小量,但这个小量又要确保远大于浮点运算结果的不确定量。eps最常见的取值是$1e-8$左右。引入eps后,我们判断两浮点数a、b相等的方式如下:
	\begin{lstlisting}[language=c++]
int sgn(double a){return a < -eps ? -1 : a < eps ? 0 : 1;}
	\end{lstlisting}
	这样,我们才能把相差非常近的浮点数判为相等;同时把确实相差较大(差值大于eps)的数判为不相等。\\
	养成好习惯,尽量不要再对浮点数做$==$判断。\\

	\subsubsection{eps带来的函数越界}
	如果sqrt(a), asin(a), acos(a) 中的a是你自己算出来并传进来的,那就得小心了。\\
	如果a本来应该是$0$的,由于浮点误差,可能实际是一个绝对值很小的负数(比如$-1e-12$),这样sqrt(a)应得$0$的,直接因a不在定义域而出错。\\
	类似地,如果a本来应该是$\pm 1$,则asin(a)、acos(a)也有可能出错。\\
	因此,对于此种函数,必需事先对a进行校正。\\
	
	\subsubsection{输出陷阱I}
	现在考虑一种情况,题目要求输出保留两位小数。有个case的正确答案的精确值是$0.005$,按理应该输出$0.01$,但你的结果可能是$0.005000000001$(恭喜),也有可能是$0.004999999999$(悲剧),如果按照\textsf{printf(“\%.2lf”, a)}输出,那你的遭遇将和括号里的字相同。\\
	解决办法是,如果$a$为正,则输出$a+eps$, 否则输出$a-eps$\\
	
	\subsubsection{输出陷阱II}
	ICPC题目输出有个不成文的规定(有时也成文),不要输出:$-0.000$\\
	那我们首先要弄清,什么时候按\textsf{printf(“\%.3lf”, a)}输出会出现这个结果。\\
	直接给出结果好了:$a\in (-0.000499999\cdots , -0.000\cdots 1)$\\
	所以,如果你发现a落在这个范围内,请直接输出$0.000$。更保险的做法是用\textsf{sprintf}直接判断输出结果是不是$-0.000$再予处理。\\

	\subsubsection{范围越界}
	请注意,虽然double可以表示的数的范围很大,却不是不穷大,上面说过最大是$1e308$。所以有些时候你得小心了,比如做连乘的时候,必要的时候要换成对数的和。\\
	
	\subsubsection{关于set}
	经观察,set不是通过$==$来判断相等的,是通过$<$来进行的,具体说来,只要$a<b$ 和$b<a$都不成立,就认为$a$和$b$相等,可以发现,如果将小于定义成:
	\begin{lstlisting}[language=c++]
bool operator < (const Dat dat)const{return val < dat.val - eps;}
	\end{lstlisting}
	就可以解决问题了。(基本类型不能重载运算符,所以封装了下)\\
	
	\subsubsection{输入值波动过大}
	这种情况不常见,不过可以帮助你更熟悉eps。假如一道题输入说,给一个浮点数$a$, $1e-20 < a < 1e20$。那你还敢用$1e-8$做eps么?合理的做法是把eps按照输入规模缩放到合适大小。\\
	
	\subsubsection{一些建议}
	容易产生较大浮点误差的函数有asin、 acos。欢迎尽量使用atan2。\\
	另外,如果数据明确说明是整数,而且范围不大的话,使用int或者long long代替double都是极佳选择,因为就不存在浮点误差了\\