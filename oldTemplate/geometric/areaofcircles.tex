\subsection{多圆面积并}
	\subsubsection{去重}
	有时候可能需要去掉不需要的圆
	\begin{lstlisting}[language=c++]
for (int i = 0; i < n; i++)
{
	scanf("%lf%lf%lf",&c[i].c.x,&c[i].c.y,&c[i].r);
	del[i] = false;
}
for (int i = 0; i < n; i++)
	if (del[i] == false)
	{
		if (c[i].r == 0.0)  del[i] = true;
		for (int j = 0; j < n; j++)
			if (i != j)
				if (del[j] == false)
					if (cmp(Point(c[i].c,c[j].c).Len()+c[i].r,c[j].r) <= 0)
						del[i] = true;
	}
tn = n;
n = 0;
for (int i = 0; i < tn; i++)
	if (del[i] == false)
		c[n++] = c[i];
	\end{lstlisting}
	
	\subsubsection{圆并}
	$ans[i]$表示被覆盖$i$次的面积
	\begin{lstlisting}[language=c++]
const double pi = acos(-1.0);
const double eps = 1e-8;
struct Point
{
	double x,y;
	Point(){}
	Point(double _x,double _y)
		{
			x = _x;
			y = _y;
		}
	double Length()
		{
			return sqrt(x*x+y*y);
		}
};
struct Circle
{
	Point c;
	double r;
};
struct Event
{
	double tim;
	int typ;
	Event(){}
	Event(double _tim,int _typ)
		{
			tim = _tim;
			typ = _typ;
		}
};

int cmp(const double& a,const double& b)
{
	if (fabs(a-b) < eps)	return 0;
	if (a < b)  return -1;
	return 1;
}

bool Eventcmp(const Event& a,const Event& b)
{
	return cmp(a.tim,b.tim) < 0;
}

double Area(double theta,double r)
{
	return 0.5*r*r*(theta-sin(theta));
}

double xmult(Point a,Point b)
{
	return a.x*b.y-a.y*b.x;
}

int n,cur,tote;
Circle c[1000];
double ans[1001],pre[1001],AB,AC,BC,theta,fai,a0,a1;
Event e[4000];
Point lab;

int main()
{
	while (scanf("%d",&n) != EOF)
	{
		for (int i = 0;i < n;i++)
			scanf("%lf%lf%lf",&c[i].c.x,&c[i].c.y,&c[i].r);
		for (int i = 1;i <= n;i++)
			ans[i] = 0.0;
		for (int i = 0;i < n;i++)
		{
			tote = 0;
			e[tote++] = Event(-pi,1);
			e[tote++] = Event(pi,-1);
			for (int j = 0;j < n;j++)
				if (j != i)
				{
					lab = Point(c[j].c.x-c[i].c.x,c[j].c.y-c[i].c.y);
					AB = lab.Length();
					AC = c[i].r;
					BC = c[j].r;
					if (cmp(AB+AC,BC) <= 0)
					{
						e[tote++] = Event(-pi,1);
						e[tote++] = Event(pi,-1);
						continue;
					}
					if (cmp(AB+BC,AC) <= 0) continue;
					if (cmp(AB,AC+BC) > 0)  continue;
					theta = atan2(lab.y,lab.x);
					fai = acos((AC*AC+AB*AB-BC*BC)/(2.0*AC*AB));
					a0 = theta-fai;
					if (cmp(a0,-pi) < 0)	a0 += 2*pi;
					a1 = theta+fai;
					if (cmp(a1,pi) > 0)	 a1 -= 2*pi;
					if (cmp(a0,a1) > 0)
					{
						e[tote++] = Event(a0,1);
						e[tote++] = Event(pi,-1);
						e[tote++] = Event(-pi,1);
						e[tote++] = Event(a1,-1);
					}
					else
					{
						e[tote++] = Event(a0,1);
						e[tote++] = Event(a1,-1);
					}
				}
			sort(e,e+tote,Eventcmp);
			cur = 0;
			for (int j = 0;j < tote;j++)
			{
				if (cur != 0 && cmp(e[j].tim,pre[cur]) != 0)
				{
					ans[cur] += Area(e[j].tim-pre[cur],c[i].r);
					ans[cur] += xmult(Point(c[i].c.x+c[i].r*cos(pre[cur]),c[i].c.y+c[i].r*sin(pre[cur])),
										Point(c[i].c.x+c[i].r*cos(e[j].tim),c[i].c.y+c[i].r*sin(e[j].tim)))/2.0;
				}
				cur += e[j].typ;
				pre[cur] = e[j].tim;
			}
		}
		for (int i = 1;i < n;i++)
			ans[i] -= ans[i+1];
		for (int i = 1;i <= n;i++)
			printf("[%d] = %.3f\n",i,ans[i]);
	}
	return 0;
}
	\end{lstlisting}