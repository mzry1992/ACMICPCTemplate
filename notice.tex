\section{注意事项}
	\Large
	\hrule\ 
	\\
	输入输出格式?调试信息?初始化?算术溢出?数组大小?\\
	\\
	左右端点范围?acos/asin/sqrt 函数定义域?精度问题?\\
	\\
	二分答案?暴力?单调性?凸性?块状结构?函数式?对偶问题?\\
	\\
	\hrule\
	\\
	排序的时候注意一下是否需要记录排序前的位置!\\
	\\
	使用map进行映射的时候,不要用下面这种不安全写法
	\begin{lstlisting}[language=c++,
    basicstyle=\sf\Large,
    numberstyle=\sf\Large,
    commentstyle=\sf\Large]
if (mp.find(s) == mp.end())
	mp[s] = mp.size()-1;//挂成狗

if (mp.find(s) == mp.end())
{
	int tmp = mp.size();
	mp[s] = tmp;//正确
}
	\end{lstlisting}
	~\\
	$10^{6}$数量级慎用后缀数组\\
	\\
	TLE的时候要冷静哟。。\\
	\\
	思考的时候结合具体步骤来的话 会体会到一些不同的东西\\
	\\
	C++与G++是很不一样的。。。\\
	\\
	map套字符串是很慢的。。。\\ 
	\\
	栈会被记录内存。。。\\
	\\
	浮点数最短路要注意取$\leq$来判断更新。。。\\
	\\
	注意 long long\\
	\\
	不要相信.size()\\
	\\
	重复利用数组时 小心数组范围\\
	\\
	先构思代码框架 每当实际拍马框架变化时 停手 重新思考\\
	\\
	有时候四边形不等式也是帮得上忙的 dp 优化是可以水的\\
	\\
	结构体里面带数组会非常慢,有时候 BFS 把数组压成数字会快很多。\\
	\\
	\begin{lstlisting}[language=c++,
    basicstyle=\sf\Large,
    numberstyle=\sf\Large,
    commentstyle=\sf\Large]
void fun(int a[])
{
	printf("%d\n",sizeof(a));
}
	\end{lstlisting}
	结果是 sizeof(a[0]),如果传数组指针然后要清空的话不要用 sizeof。\\
	\\
	sqrt 某些时候会出现 sqrt(-0.00)的问题。\\
	\\
	将code::blocks的默认终端改成gnome-terminal
	\begin{lstlisting}[language=c++,
    basicstyle=\sf\Large,
    numberstyle=\sf\Large,
    commentstyle=\sf\Large]
gnome-terminal -t $TITLE -x
	\end{lstlisting}
	~\\
	最小割割集找法 在残量网络中从源点出发能到的点集记为S原图中S到S’的边即是最小割集\\
	\\
	double全局变量初始值可能不是$0$\\
	\normalsize