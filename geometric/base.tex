\subsection{基本函数}
	\subsubsection{Point定义}
		\begin{lstlisting}[language=c++]
struct Point
{
	double x, y;
	Point() {}
	Point(double _x, double _y)
	{
		x = _x, y = _y;
	}
	Point operator -(const Point &b)const
	{
		return Point(x-b.x, y-b.y);
	}
	double operator *(const Point &b)const
	{
		return x*b.y-y*b.x;
	}
	double operator &(const Point &b)const
	{
		return x*b.x+y*b.y;
	}
	void transXY(double B)
	{
		double tx = x,ty = y;
		x = tx*cos(B)-ty*sin(B);
		y = tx*sin(B)+ty*cos(B);
	}
};
		\end{lstlisting}
		
	\subsubsection{Line定义}
		\begin{lstlisting}[language=c++]
struct Line
{
	Point s, e;
	double k;
	Line() {}
	Line(Point _s, Point _e)
	{
		s = _s, e = _e;
		k = atan2(e.y-s.y, e.x-s.x);
	}
	Point operator &(const Line &b)const
	{
		Point res = s;
		//注意:有些题目可能会有直线相交或者重合情况
		//可以把返回值改成`pair<Point,int>`来返回两直线的状态。
		double t = ((s-b.s)*(b.s-b.e))/((s-e)*(b.s-b.e));
		res.x += (e.x-s.x)*t;
		res.y += (e.y-s.y)*t;
		return res;
	}
};
		\end{lstlisting}
		
	\subsubsection{距离:点到直线距离}
	result:点到直线最近点
	\begin{lstlisting}[language=c++]
Point NPT(Point P, Line L)  
{ 
	Point result; 
	double a, b, t; 

	a = L.e.x-L.s.x; 
	b = L.e.y-L.s.y; 
	t = ((P.x-L.s.x)*a+(P.y-L.s.y)*b)/(a*a+b*b); 

	result.x = L.s.x+a*t; 
	result.y = L.s.y+b*t; 
	return dist(P, result); 
}
	\end{lstlisting}
	
	\subsubsection{距离:点到线段距离}
		res:点到线段最近点
		\begin{lstlisting}[language=c++]
Point NearestPointToLineSeg(Point P, Line L)
{
	Point result;
	double a, b, t;

	a = L.e.x-L.s.x;
	b = L.e.y-L.s.y;
	t = ( (P.x-L.s.x)*a+(P.y-L.s.y)*b )/(a*a+b*b);

	if (t >= 0 && t <= 1)
	{
		result.x = L.s.x+a*t;
		result.y = L.s.y+b*t;
	}
	else
	{
		if (dist(P,L.s) < dist(P,L.e))
			result = L.s;
		else
			result = L.e;
	}
	return result;
}
		\end{lstlisting}
		旧版
		\begin{lstlisting}[language=c++]
double CalcDis(Point a,Point s,Point e) //点到线段距离
{
	if (sgn((e-s)*(a-s)) < 0 || sgn((s-e)*(a-e)) < 0)
		return min(dist(a,s),dist(a,e));
	return abs(((s-a)*(e-a))/dist(s-e));
}
		\end{lstlisting}
		
	\subsubsection{面积:多边形}
		点按逆时针排序。
		\begin{lstlisting}[language=c++]
double CalcArea(Point p[], int n)
{
	double res = 0;
	for (int i = 0; i < n; i++)
		res += (p[i]*p[(i+1) % n])/2;
	return res;
}
		\end{lstlisting}
	
	\subsubsection{判断:线段相交}
		\begin{lstlisting}[language=c++]
bool inter(Line l1,Line l2)
{
	return 
	max(l1.s.x,l1.e.x) >= min(l2.s.x,l2.e.x) &&
	max(l2.s.x,l2.e.x) >= min(l1.s.x,l1.e.x) &&
	max(l1.s.y,l1.e.y) >= min(l2.s.y,l2.e.y) &&
	max(l2.s.y,l2.e.y) >= min(l1.s.y,l1.e.y) &&
	sgn((l2.s-l1.s)*(l1.e-l1.s))*sgn((l2.e-l1.s)*(l1.e-l1.s)) <= 0 &&
	sgn((l1.s-l2.s)*(l2.e-l2.s))*sgn((l1.e-l2.s)*(l2.e-l2.s)) <= 0;
}
		\end{lstlisting}
	
	\subsubsection{判断:点在线段上}
		\begin{lstlisting}[language=c++]
bool OnSeg(Line a,Point b)
{
	return ((a.s-b)*(a.e-b) == 0 &&
			(b.x-a.s.x)*(b.x-a.e.x) <= 0 &&
			(b.y-a.s.y)*(b.y-a.e.y) <= 0);
}
		\end{lstlisting}
		
	\subsubsection{判断:点在多边形内}
		凸包且按逆时针排序
		\begin{lstlisting}[language=c++]
bool inPoly(Point a,Point p[],int n)
{
	for (int i = 0;i < n;i++)
		if ((p[i]-a)*(p[(i+1)%n]-a) < 0)
			return false;
	return true;
}
		\end{lstlisting}
		射线法, 多边形可以是凸的或凹的\\
		poly的顶点数目要大于等于3\\
		返回值为:\\
		0  -- 点在poly内\\
		1  -- 点在poly边界上\\
		2  -- 点在poly外
		\begin{lstlisting}[language=c++]
int inPoly(Point p,Point poly[], int n)
{
	int i, count;
	Line ray, side;

	count = 0;
	ray.s	= p;
	ray.e.y  = p.y;
	ray.e.x  = -1;//`-INF,注意取值防止越界!`

	for (i = 0; i < n; i++)
	{
		side.s = poly[i];
		side.e = poly[(i+1)%n];

		if(OnSeg(p, side))
			return 1;

		// 如果side平行x轴则不作考虑
		if (side.s.y == side.e.y)
			continue;

		if (OnSeg(side.s, ray))
		{
			if (side.s.y > side.e.y) count++;
		}
		else if (OnSeg(side.e, ray))
		{
			if (side.e.y > side.s.y) count++;
		}
		else if (inter(ray, side))
		{
			count++;
		}
	}
	return ((count % 2 == 1) ? 0 : 2);
}
		\end{lstlisting}
		
	\subsubsection{判断:两凸包相交}
		需要考虑这几个:一个凸包的点在另外一个图包内(包括边界);一个凸包的某条边与另一个凸包某条边相交;如果凸包可能退化成点线还需要判断点在线段上和点和点重合。
		
	\subsubsection{排序:叉积极角排序}
	\begin{lstlisting}[language=c++]
bool cmp(const Point& a,const Point& b)
{
	if (a.y*b.y <= 0)
	{
		if (a.y > 0 || b.y > 0)	return a.y < b.y;
		if (a.y == 0 && b.y == 0)	return a.x < b.x;
	}
	return a*b > 0;
}
	\end{lstlisting}